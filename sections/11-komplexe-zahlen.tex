%!TEX root = ../basiskurs.tex
\section{Die komplexen Zahlen}

\subsection{Literatur}
\begin{enumerate}
\item Dittmann, Komplexe Zahlen, bsv Mathematik, 1973-1995
\item Nierdrenk-Felgner, LS Komplexe Zahlen, Klett Verlag 2004
\end{enumerate}

\subsection{Definition}
\begin{itemize}
	\item Wir führen auf $\mathbb{R}^2$ eine Multiplikation ein durch folgende Regel $(a,b) \cdot (c,d) = (ac-bd, bc+ad)$
	Wie man leicht nachprüft, erhält man damit einen Körper $\mathbb{C}$.
	Der Körper $\mathbb{C}$ enthält $\mathbb{R}$ mittels der injektiven Abbildung $\iota: \mathbb{C} \rightarrow \mathbb{R}: a \rightarrowtail (a,0)$.
	Dabei ist $\iota$ mit $+$ und $\cdot$ verträglich.

	\item Schreibweisen: Statt $e_1 = (1,0)$ schreibe 1, statt $e_2 = (0,1)$ schreibe $i$.
	Jedes Element von $\mathbb{C}$ hat also eine eindeutige Darstellung der Form $a + b * i$ mit $a, b \in \mathbb{R}$.

	\item Für $z = a+b*i \in \mathbb{C}$ heißt $Re(z)=a$ der Realteil und $Im(z)=b$ der Imaginärteil.
\end{itemize}

\subsection{Bemerkung}
\begin{enumerate}
	\item Die Zahl $i$ erfüllt $i^2 = -1$, denn $(0,1) \cdot (0,1) = (-1,0)$.
	Man schreibt daher auch $i = \sqrt{-1}$.

	\item Für $a \in \mathbb{R}_{\geq 0}$ gilt $(\sqrt{a} * i)^2 = a * (-1) = -a$, also $\pm \sqrt{a} * i = \sqrt{-a}$.

	\item Für $z = a+bi \in \mathbb{C} \setminus \{0\}$ gilt
	\[\frac{1}{z} = \frac{1}{a+bi} = \frac{a-bi}{(a+bi)(a-bi)} = \frac{a-bi}{a^2+b^2} = \frac{a}{a^2+b^2} - \frac{b}{a^2+b^2} * i\]
\end{enumerate}

\subsection{Beispiele}
\begin{enumerate}
	\item Für $n \in \mathbb{Z}$ gilt
	\[i^n = \begin{cases}
	i,  &\text{ falls } n \equiv 1 \text{ (mod 4)}\\
	-1, &\text{ falls } n \equiv 2 \text{ (mod 4)}\\
	-i, &\text{ falls } n \equiv 3 \text{ (mod 4)}\\
	1,  &\text{ falls } n \equiv 0 \text{ (mod 4)}
	\end{cases} \]

	\item In $\mathbb{C}$ besitzt jede quadratische Gleichung genau 2 Lösungen (Mitternachtsformel).
\end{enumerate}

\subsection{Definition}
Die Abbildung $\kappa: \mathbb{C} \rightarrow \mathbb{C}: a+b \rightarrowtail a-bi$ heißt die komplexe Konjugation.
Für $z \in \mathbb{C}$ schreiben wir statt $\kappa(z)$ auch $\overline{z}$.

\subsection{Bemerkung (Rechenregeln für die komplexe Konjugation)}
\label{sec:regeln-komplexe}
Für $z, z_1, z_2 \in \mathbb{C}$ gilt:
\begin{enumerate}
\item $\overline{z} = z$
\item $\overline{z_1 + z_2} = \overline{z_1} + \overline{z_2}$
\item $\overline{z_1 \cdot z_2} = \overline{z_1} \cdot \overline{z_2}$
\item Für alle $a \in \mathbb{R}$ gilt $\overline{a \cdot z} = a \cdot \overline{z}$
\item $\frac{1}{\overline{z}} = \overline{\frac{1}{z}}$ falls $z \neq 0$
\item
	\begin{align}
	z + \overline{z} &= 2 a &= 2 * Re(z)\\
	z - \overline{z} &= 2 bi &= 2 * Im(z) * i
	\begin{cases}
	Re(z) = \frac{1}{2}(z + \overline{z}) \\
	Im(z) = \frac{1}{2i}(z - \overline{z}) = -\frac{i}{2}(z - \overline{z})
	\end{cases}
	\end{align}
\end{enumerate}

% </2017-01-25> <2017-02-01>

\subsection{Bemerkung}
$\mathbb{C}$ ist ein 2-dimensionaler $\mathbb{R}$-Vektorraum mit Basis $\{1,i\}$.
d.h. jede komplexe Zahl $i \in \mathbb{C}$ hat eine eindeutige Darstellung $z = a + b * i$ mit  $a,b \in \mathbb{R}$ und $i \in \mathbb{C}$.
Es gilt $i^2 = 1$. Siehe auch \ref{sec:regeln-komplexe}.


\subsection{Bemerkung}
In $\mathbb{A}^2(\mathbb{R})$ führen wir kartesische Koordinaten ein und identifizieren die Zahl $z = a+b*i \in \mathbb{C}$ mit dem Punkt $(a,b)$.
\todo{Skizze}

Geometrische Interpretation der Körperoperationen
\begin{enumerate}
\item Addition $(a+b*i) + (c+d*i) = (a+c) + (b+d) * i$ entspricht der Vektoraddition. \todo{Skizze}
\item Multiplikation mit $i$, $(a+b*i) * i = -b + a * i$, entspricht der Drehung um $90 ^\circ$ um den Nullpunkt im mathematisch positiven Sinne (gegen den Uhrzeigersinn). \todo{Skizze}
\item Abstand zum Nullpunkt / Länge des Vektors:\\
	Der Betrag einer komplexen Zahl ist $z = a + b * i$ ist $|z| = \sqrt{a^2 + b^2}$. (nach dem Satz von Pythagoras) \todo{Skizze}\\
	Eigenschaften des Betrags:
	\begin{itemize}
	\item $|z_1 * z_2| = |z_1| * |z_2|$ für $z_1, z_2 \in \mathbb{C}$
	\item $|c * z| = c * |z|$ für $c \in \mathbb{R}$
	\item $|z_1 + z_2| \leq |z_1| + |z_2|$ für $z_1, z_2 \in \mathbb{C}$ (Dreiecksungleichung)
	\end{itemize}
\item Der Winkel $\varphi$ (oder das Argument) einer komplexen Zahl $z \in \mathbb{Z} \setminus \{0\}$ mit $z = a+b*i$ und $a,b \in \mathbb{R}$ erfüllt $z = |z| * \cos{\varphi} + |z| * \sin{\varphi} * i$ und $\varphi \in [0, 2 \pi[$.
	Wir schreiben $\varphi = arc (z)$ (''arcus`` \^= Bogen) . \todo{Skizze}
\item Sei $r=|z|$ und $\varphi = arc(z)$.
	Dann ist die Multiplikation mit $z = r * \cos{\varphi} + r * \sin{\varphi} * i = r * (\cos{\varphi} + \sin{\varphi} * i)$ die Komposition der Multiplikation mit $\cos{\varphi} + \sin{\varphi} * i$ und der Multiplikation mit $r \in \mathbb{R}_+$.
	Letztere ist die zentrische Streckung um den Faktor $r$ mit Mittelpunkt 0.
	Nun wende die erste Multiplikation an auf $\tilde{z} = \tilde{r} * (\cos{\psi} + \sin{\psi} * i)$ .
	Dann gilt
	\begin{align*}
	\tilde{z} * (\cos{\psi} + \sin{\psi} * i) &=
	\tilde{r} * (\cos{\psi} + \sin{\psi} * i) * (\cos{\varphi} + \sin{\varphi} * i) \\ &=
	\tilde{r} * \left[ (\cos{\psi}\cos{\varphi} - \sin{\psi}\sin{\varphi}) + (\sin{\psi}\cos{\varphi} + \cos{\psi}\sin{\varphi}) * i \right] \\ &=
	\tilde{r} * (\cos{(\psi + \varphi)} + \sin{(\psi + \varphi)} * i)
	\end{align*}
	Das Produkt hat also den selben Betrag, aber der Winkel ist um $\varphi$ größer.
	Die Multiplikation mit $\cos{\varphi} + \sin{\varphi} * i$ entspricht also der Drehung um den Winkel $\varphi$ um den Mittelpunkt.
	Insgesamt ist die Multiplikation mit $z$ also eine Drehstreckung mit Zentrum 0.
\end{enumerate}

\subsection{Bemerkung (Algebraische Beschreibung geometrischer Mengen)}
\begin{enumerate}
\item (Kreis mit Mittelpunkt $m \in \mathbb{C}$ und Radius $r \in \mathbb{R}_+$)\\
	Der Abstand von $z \in \mathbb{C}$ zu $m$ ist gegeben durch $|z - m|$.
	Der Kreis ist also gegeben durch
	\[K = \left\{ z \in \mathbb{C} \middle| |z - m| = r \right\} \]
\item Die Gerade $G$ durch $z = a + b * i$ und $\tilde{z} = c + d * i$ ist gegeben durch
	\[G = \left\{ \lambda \in \mathbb{R} \middle| z + \lambda (\tilde{z} - z) \right\} \]
	Implizite Darstellung  mit $e \in \mathbb{C}, f \in \mathbb{R}$:
	\[G = \left\{ z \in \mathbb{C} \middle| \overline{e} z + e \overline{z} + f = 0 \right\} \]
	Es gilt: $\overline{e} z + e \overline{z} = \overline{e} z + \overline{(\overline{e} z)} = 2 Re (\overline{e} z)$.
	Schreibe $e = \alpha + \beta i$ mit $\alpha, \beta \in \mathbb{R}$ und $z = x + y * i$ mit $x, y \in \mathbb{R}$.
	Dann folgt:
	\[Re(\overline{e} z) = Re ((\alpha - \beta i)(x + y i))
	= Re ((\alpha x + \beta y) + (-\beta x + \alpha y)i)
	= \alpha x + \beta y
	\left( = \left< \binom{\alpha}{\beta}, \binom{x}{y} \right> \right) \]
	Dann folgt: $\overline{e} z + e \overline {z} + \underbrace{f}_{\in \mathbb{R}} = \underbrace{2 \alpha x + 2 \beta y + f} _{\text{implizite Geradengleichung in }\mathbb{R}^2} = 0$
\end{enumerate}

\subsection{Bemerkung (Algebraische Interpretation von Abbildungen der Zeichenebene)}
\begin{enumerate}
\item Die Translation $\tau$ um den Vektor $a + b * i$ ist die Addition
	$\tau: \mathbb{C} \rightarrow \mathbb{C}; z \rightarrowtail z + (a + b * i)$.
\item Die Drehung um den Winkel $\varphi$ um den Nullpunkt entspricht der Multiplikation mit $\cos{\varphi} + \sin{\varphi} * i$.
	Sei nun $m \in \mathbb{C}$.
	Die Drehung $\varrho_{m, \varphi}$ ist die Komposition von
	\begin{enumerate}
	\item Verschiebung um $-m$.
	\item Drehung um 0 um den Winkel $\varphi$.
	\item Verschiebung um $m$.
	\end{enumerate}
	Es gilt also $\varrho_{m, \varphi}: \mathbb{C} \rightarrow \mathbb{C}; z \rightarrowtail (\cos{\varphi} + \sin{\varphi} * i) (z-m) + m$.
	\todo{Skizze}
\item Die Spiegelung an einer Geraden durch 0.
	$G$ habe den Richtungsvektor $a + b *i$.
	Die Gerade durch $z$ und $\tilde{z}$ hat dann den Richtungsvektor $-b + a * i$.
	Dann ist $G \cap H$ der Lotfußpunkt $l \in \mathbb{C}$ und es gilt $\tilde{z} = \sigma_G(z) = z + 2 (l - z) = -z + 2l$.
	\todo{Skizze}
\end{enumerate}

\subsection{Bemerkung (Geometrische Interpretation von $z \rightarrowtail \overline{z}$)}
Die Abbildung $\kappa: \mathbb{C} \rightarrow \mathbb{C}; a+b*i \rightarrowtail a-b*i$ entspricht der Spieglung an der $Re(z)$-Achse (x-Achse).
\todo{Skizze}

\subsection{Bemerkung (Inversion am Kreis)}
Die Abbildung $\iota: \mathbb{C} \setminus \{0\} \rightarrow \mathbb{C} \setminus \{0\}; z \rightarrowtail \frac{1}{\overline{z}}$
wirkt auf $z = a+b*i = r*(\cos{\varphi} + \sin{\varphi} i)$ wie folgt
\begin{align*}
\iota(z) &= \frac{1}{\overline{z}} \\
 &= \overline{\frac{1}{r} * \frac{1}{\cos{\varphi}+\sin{\varphi}*i}} \\
 &= \overline{\frac{1}{r} * \frac{1 * (\cos{\varphi}-\sin{\varphi}*i)}{(\cos{\varphi}+\sin{\varphi}*i)*(\cos{\varphi}-\sin{\varphi}*i)}} \\
 &= \frac{1}{r} * \overline{(\cos{\varphi}-\sin{\varphi}*i)} \\
 &= \frac{1}{r} * (\cos{\varphi}+\sin{\varphi}*i)
\end{align*}
Die Zahl $\iota(z)$ liegt also auf dem gleichen Halbstrahl durch 0 wie $z$.
Ihr Betrag ist $\frac{1}{r}$.
Der Einheitskreis $\mathbb{E} = \left\{ \cos{\varphi} + \sin{\varphi}*i \middle| \varphi \in [0, 2\pi[ \right\}$ bleibt dabei fest.
Es gilt $\iota^2(z) = \iota(\iota(z)) = z$.
Die geometrische Interpretation von $\frac{1}{z}$ ist also der Punkt, der aus $z$ entsteht indem man zuerst an der $Re(z)$-Achse spiegelt und dann die Inversion am Einheitskreis durchführt.
\[\frac{1}{z} = \iota (\overline{z}) = \frac{1}{\overline{(\overline{z})}} = \frac{1}{z}\]
\todo{Skizze}

\subsection{Satz (Eigenschaften der Inversion am Kreis)}
\begin{enumerate}
\item Ist $G$ eine Gerade in $\mathbb{C}$, die nicht durch 0 geht, so ist $\iota(G)$ ein Kreis.
\item Ist $G$ eine Gerade in $\mathbb{C}$, die durch 0 geht, so ist $\iota(G \setminus \{0\}) = G \setminus \{0\}$.
\item Ist $K$ ein Kreis in $\mathbb{C}$, der durch 0 geht, so ist $\iota(K \setminus \{0\})$ eine Gerade, die nicht durch 0 geht.
\item Ist $K$ ein Kreis in $\mathbb{C}$, der nicht durch 0 geht, so ist $\iota(K)$ wieder ein Kreis (der nicht durch 0 geht).
\end{enumerate}

\subsection{Bemerkung (Die komplexe e-Funktion)}
Die reelle e-Funktion $x \rightarrowtail e^x$ erfüllt $(e^x)' = e^x$ und $e^0=1$.
Für alle $x \in \mathbb{R}$ gilt $e^x = 1+\frac{1}{1!}x+\frac{1}{2!}x^2+\frac{1}{3!}x^3+\cdots = \sum_{n \geq 0} \frac{1}{n!}x^n$.
Für $z \in \mathbb{C}$ konvergiert die Reihe $e^z = \sum_{n \geq 0} \frac{1}{n!}z^n \in \mathbb{C}$ ebenfalls.
Man prüft nach, dass $e^{z_1+z_2} = e^{z_1} * e^{z_2}$ für alle $z_1, z_2 \in \mathbb{C}$ gilt.

% </2017-02-01>