%!TEX root = ../basiskurs.tex
\section{Kombinatorik}
Kombinatorik ist die Kunst des Zählens. {\small \textit{Analysis ist alleine schon per Defition \textbf{vüüüll} schwarer als Kombinatorik.}}

\subsection{Definition}
Seien $a_1,\ldots,a_n$ mit $n\geq1$ paarweise verschiedene Objekte.
\begin{enumerate}
\item Eine feste Anordnung $(a_{i_1},a_{i_2},a_{i_3},\ldots,a_{i_n})$ mit $\{i_1,\ldots,i_n\} = \{1,\ldots,n\}$ heißt auch eine Permutation von $a_1,\ldots,a_n$. \\
\textbf{Schreibweisen}: $\sigma = \binom{a_1,\ldots,a_n}{a_{i_1},\ldots,a_{i_n}}$ oder einfach $\sigma = \binom{1,2,\ldots,n}{i_1,i_2,\ldots,i_n}$.\\
Ohne Einschränkung betrachten wir also meist die Permutation der Menge $\{1,2,\ldots,n\}$.
\item Die Menge aller Permutationen von $n$ Objekten heißt die  symmetrische Gruppe $S_n$.
\end{enumerate}

\subsection{Satz}
Die Gruppe $S_n$ hat $n!$ Elemente.
\textbf{Beweis:} Halte ein Element, z.B. $a_1$ fest. Für die Bilder $\sigma(a_1)$ unter $\sigma\in S_n$ gibt es $n$ Möglichkeiten.
Für $\sigma(a_2)$ gibt es dann noch $n-1$ Möglichkeiten.
Für $\sigma(a_3)$ gibt es dann noch $n-2$ Möglichkeiten. \dots
Am Ende gibt es dann nur noch $1$ Auswahl für $\sigma(a_n)$.
Insgesamt gibt es $n * (n-1) * (n-2) * \cdots * 2 * 1 = n!$ Permutationen.

\subsection{Beispiel}
Wir ordnen Permutationen.
Wie viele Transpositionen (Permutationen der Form $\tau=\binom{1 \cdots i \cdots j \cdots n}{1 \cdots j \cdots i \cdots n}$) braucht man, um eine beliebige Permutation rückgängig zu machen?
\begin{enumerate}
\item Obere Schranke:
	\begin{enumerate}
	\item Bringe zuerst die $1$ an die erste Stelle.
	\item Bringe dann die $2$ an die zweite Stelle.
	\item \dots
	\item Bringe zuletzt die $n-1$ an die vorletzte Stelle. \\
		Die Zahl $n$ ist damit automatisch richtigerweise an der letzten Stelle.
	\end{enumerate}
	Somit braucht man max. $n-1$ Transpositionen.
\item Kandidat fürs Maximum / Untere Schranke:
	\[\sigma = \left( \begin{matrix}
	1 & 2 & 3 & \cdots & (n-1) & n \\ n & (n-1) & (n-2) & \cdots & 2 & 1
	\end{matrix} \right) \]
	Man braucht $\tau_{1n}, \tau_{2(n-1)}, \tau_{3(n-2)}, \ldots \tau_{\frac{n-1}{2} \frac{n+1}{2}}$, falls $n$ ungerade ist. \\
	Für $n$ gerade braucht man $\tau_{1n}, \tau_{2(n-1)}, \tau_{3(n-2)}, \ldots \tau_{\frac{n}{2} \frac{n+2}{2}}$. \\
	Die sind $\left \lceil \frac{n-1}{2} \right \rceil = \left \lfloor \frac{n+1}{2} \right \rfloor $ Stück.
\item Aus der Zykelzerlegung folgt, dass die untere Schranke konstant ist.
	\[\text{Zykel: } \sigma = \underbrace{\left( \begin{matrix}
	1 & 2 & 3 & 4 & 5 \\ 2 & 3 & 4 & 5 & 1
	\end{matrix} \right)}_{5-Zykel} \text{ oder }
	\sigma = \underbrace{\left( \begin{matrix}
	1 & 2 & 3 & 4 & 5 \\ 1 & 2 & 4 & 5 & 3
	\end{matrix} \right)}_{3-Zykel} \]
	Die Darstellung eines k-Zykels kann man mit einer Transposition in zwei kürzere Zyklen aufspalten.
	Ist $m$ die Zahl der Zyklen in der Zykelzerlegung so braucht man max $n-m$ Transposition um $\sigma$ in lauter Transpositionen aufspalten.
\item Aus den beiden vorherigen Punkten folgt, dass das gesuchte Maximum $\left \lfloor \frac{n+1}{2} \right \rfloor$ ist.
\end{enumerate}

\subsection{Defintion (Teilmengen zählen)}
Gegeben seien $n$ paarweise verschiedene Objekte $a_1,\ldots,a_n$.
\begin{itemize}
\item Sei $0 \leq m \leq n$. Eine Teilmenge von der Menge $\{a_1,\ldots,a_n\}$ bestehend aus $m$ Elemente heißt auch eine Auswahl von $m$ Elementen.
\item Die Anzahl der Auswahlen von $m$ Elementen aus $\{a_1,\ldots,a_n\}$ heißt der Binomialkoeffizient $\binom{n}{m}$.
	(sprich ``m aus n'', ``n über m'' oder ``n choose m'')
\end{itemize}

\subsection{Satz (Formel für den Binomialkoeffizienten)}
Für $n \geq 1$ und $0 \leq m \leq n$ gilt mit $0!=1$:
\[ \binom{n}{m} = \frac{n*(n-1)*(n-2)*\cdots*(n-m+1)}{1*2*\cdots*m} = \frac{n!}{m!*(n-m)!} \]

\subsection{Bemerkung (Das paskalsche Dreieck)}
\begin{itemize}
\item Die Binomialkoeffizienten erfüllen die Formel
	\[ \binom{n}{m} = \binom{n-1}{m} + \binom{n-1}{m-1} \text{ f\"ur } m\geq1,\,n\geq2 \]
	\textbf{Beweis}: Sei $a_1$ ein festes Element von $\{a_1,\ldots,a_n\}$.
	\begin{itemize}
	\item Eine Teilmenge, die $a_1$ enthält, besteht aus $a_1$ und $(m-1)$ Elementen von $\{a_2,\ldots,a_n\}$.
		\[\rightarrow \binom{n-1}{m-1} \text{ M\"oglichkeiten}\]
	\item Eine Teilmenge, die $a_1$ nicht enthält, entspricht einer Auswahl von Elementen aus $\{a_2,\ldots,a_n\}$.
		\[\rightarrow \binom{n-1}{m} \text{ M\"oglichkeiten}\]
	\end{itemize}
	Da immer genau einer der beiden Fälle eintritt, folgt in Summe die Behauptung.
\item Die Binomialkoeffizienten sind gegeben durch \todo{Skizze paskalsches Dreieck}
\end{itemize}

\subsection{Beispiel}
Betrachte das Gitter ...

% </2017-02-08 11:21>